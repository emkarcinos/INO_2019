\documentclass{article}
\usepackage{polski}
\usepackage[utf8]{inputenc}
\usepackage{hyperref}
\hypersetup{
    colorlinks,
    citecolor=black,
    filecolor=black,
    linkcolor=black,
    urlcolor=black
}

\title{Raport z testów Gita na WMI}
\author{Marcin Kostrzewski}
\date{18 Listopad, 2019r}

\begin{document}

\maketitle
\newpage

\section{Wyniki testów}

\subsection{Logowanie do serwisu poprzez przeglądarkę.}
Oczkiwany rezulat został osiągnięty. Logowanie jest proste, przy podawaniu poprawnych danych zostajemy pomyślnie zalogowani.

\subsection{Zmiana obrazku avatara użytkownika.}
Rezultaty zostały osiągnięte. Proces był bardzo prosty, mogę wybrać dowolny avatar, niekwadratowy awatar został automatycznie wykadrowany.

\subsection{Aktualizacja swojej lokalizacji.}
Testy przebiegł pomyślnie, lecz mogę wpisać dowolną, nawet niepoprawną lokalizację.

\subsection{Aktualizacja swojego imienia i nazwiska.}
Spełniony został oczekiwany rezultat, nawet wieloczłonowe imiona i nazwiska rejestrowane są poprawnie.

\subsection{Aktualizacja swojego maila.}
Test przeszedł poprawnie, nie możemy podać maila nie zawierającego znaku @. 

\section{Rezultat}
Wszystkie testy były proste do zrealizowania, nie napotkałem żadnych problemów przy wykonywaniu ich.
Z serwisu korzysta się łatwo i dostęp do wszystkich funkcji nie jest utrudniony.

\end{document}
