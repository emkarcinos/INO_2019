\documentclass{article}
\usepackage{polski}
\usepackage[utf8]{inputenc}
\usepackage{hyperref}
\hypersetup{
    colorlinks,
    citecolor=black,
    filecolor=black,
    linkcolor=black,
    urlcolor=black
}

\title{Strategia testowania dla uczelnianego serwisu Git}
\author{Marcin Kostrzewski}
\date{13 Listopad, 2019r}

\begin{document}

\maketitle
\newpage
\tableofcontents
\newpage

\section{Produkt}
Dokument skupia się na testowaniu serwera \textbf{Git} na Wydziale Matematyki i Informatyki Uniwersytetu im. Adama Mickiewicza w Poznaniu.
Sewis dostępny jest pod adresem \url{https://git.wmi.amu.edu.pl/explore/repos}.


\section{Strategia testowania}
Strategią testowania będzie \textbf{testowanie funkcjonalne}. 
Wybrałem ją, ponieważ serwis nie oferuje dostępu do kodu źródłowego, aplikacja jest już opakowana i skompilowana.
Mamy do niej dostęp z poziomu wiersza polecenia (dostęp do repozytoriów przez aplikację \textit{git})
 oraz za pomocą interfejsu WEB. Skupimy się na testowaniu
działania całego programu, sprawdzając, czy poszczególne scenariusze użytkowania przynoszą oczekiwane
rezultaty.

\section{Zasoby testowe}
\subsection{Środowisko testowe}
Wydziałowy Git oferuje dostep zarówno w sieci wydziałowej LAN, oraz spoza wydziału.
Testować będziemy zarówno interfejs WEB z poziomu przeglądarki internetowej, oraz 
interakcje z serwerem za pomocą aplikacji terminalowej git.

\subsection{Zespół wykonawców}
Testy będą przeprowadzane przez studentów wydziału, indywidualnie. Każdy z testerów
ma we własnym zakresie sporządzić raport powstały na bazie testów z tego dokumentu.

\subsection{Warunki początkowe}
Każdy tester powinien mieć zainstalowaną dowolną przeglądarkę internetową, o wersjach nowszych niż z roku 2010.
Dodatkowo, testerzy będą korzystać z aplikacji \textit{git} na dowolnej platformie (Windows, Linux, MacOS).
\newline
Testerzy jako studenci wydziału mają utworzone konta w serwisie i znają hasła dostępu.

\newpage
\section{Scenariusze testów}

\subsection{Logowanie do serwisu z poziomu przeglądarki}
\subsubsection*{Testy:}
\begin{itemize}
    \item Wejście na stronę \url{https://git.wmi.amu.edu.pl/explore/repos} przez przeglądarkę internetową
    \item Kliknięcie w przycisk \textbf{Sign in} \textit{(Zaloguj się)} w prawym górnym rogu strony
    \item Wypełnienie danych: nazwa użytkownika oraz hasło
    \item Kliknięcie przycisku \textbf{Sign in} \textit{(Zaloguj się)}
\end{itemize}
\subsubsection*{Oczekiwany rezultat}
Użytkownik zostaje zalogowany do serwisu. Przeglądarka przekierowuje zalogowanego użytkownika do
podstrony odwiedzonej przed kliknięciem przycisku \textit{Sign in}. Przysik do zalogowania znika.

\subsection{Tworzenie repozytorium}
\textit{Użytkownik jest już zalogowany i znajduje się na stronie głównej serwisu.}
\subsubsection*{Testy:}
\begin{itemize}
    \item Kliknięcie w przycisk \texttt{+} i wybranie \textit{New repository} \textit{(Nowe repozytorium)} z listy.
    \item Wypełnienie danych: obowiązkowa nazwa repozytorium (bez znaków białych)
    \item Kliknięcie przyciku \textbf{Create Repository} \textit{(Utwórz repozytorium)}
\end{itemize}

\subsubsection*{Oczekiwany rezultat}
Repozytorium zostało utworzone w serwisie. Posiada podaną w formularzu nazwę oraz resztę informacji.
Po wejściu w repozytoria użytkownika, stworzone nowe repozytorium jest widoczne na liście.

\newpage

\subsection{Dodawanie lokalnych zmian do repozytorium}
\textit{Poniższe kroki będą bazowały na repozytorium utworzonym w poprzednim teście. Zmiany w repozytorium dokonywane będą z poziomu wiersza polecenia.
Testy odbywają się w systemie operacyjnym użytkownika.}
\subsubsection*{Testy:}
\begin{itemize}
    \item Sklonowanie repozytorium za pomocą komendy \textit{git clone URLREPOZYTORIUM}, gdzie podajemy link do stworzonego wcześniej repozytorium
    \item Wejście do folderu ze sklonowanym repozytorium, utworzenie pustego pliku testowego (np. \textit{test.txt})
    \item W terminalu w bieżącym folderze, użycie komendy \textit{git add *}, następnie \textit{git commit -m "inital commit"}, \textit{git push}
    \item Po zapytaniu o nazwę użytkownika i hasło, podanie danych dostępowe do serwisu
\end{itemize}

\subsubsection*{Oczekiwany rezultat}
Po wejściu do repozytorium z poziomu przeglądarki widzimy dodany plik w zawartości repozytorium z nazwą commita "initial commit"

\subsection{Wysyłanie pliku do repozytorium przez interfejs WEB}
\textit{Poniższe kroki testowe zakładają, że zostały wykonane polecenia z testów powyżej}
\subsubsection*{Testy:}
\begin{itemize}
    \item Wejście na stronę \url{https://git.wmi.amu.edu.pl/explore/repos} przez przeglądarkę internetową
    \item W prawym górnym rogu strony kliknięcie na avatar naszego użytkownika
    \item Wybór \textit{Your Profile} \textit{(Twój profil)} z rozwiniętej listy
    \item Kliknięcie na nazwę uprzednio utworzonego repozytorium
    \item Kliknięcie w przycisk \textbf{Upload file} \textit{(Załaduj plik)}
    \item Wybieramy dowolny plik na naszym komputerze i "przeciągamy go" na stronę w przeglądarce
    \item Wypełniamy pola ze zmianami i klikamy \textbf{Commit Changes} \textit{(Zatwierdź zmiany)}
\end{itemize}
\subsubsection*{Oczekiwany rezultat}
W repozytorium pojawił się przesłany plik z załączoną treścią commita

\newpage
\subsection{Usuwanie repozytorium}
\textit{Poniższe kroki testowe zakładają, że zostały wykonane polecenia z testów powyżej}
\subsubsection*{Testy:}
\begin{itemize}
    \item Wejście na stronę \url{https://git.wmi.amu.edu.pl/explore/repos} przez przeglądarkę internetową
    \item W prawym górnym rogu strony kliknięcie na avatar naszego użytkownika
    \item Wybór \textit{Your Profile} \textit{(Twój profil)} z rozwiniętej listy
    \item Kliknięcie na nazwę uprzednio utworzonego repozytorium
    \item Klinięcie w przycisk \textbf{Settings} \textit{(Ustawienia)}
    \item Na samym dole na liście \textbf{Danger Zone} \textit{Strefa niebezpieczeństwa} kliknięcie w przycisk \textbf{Delete This Repository} \textit{(Usuń to repozytorium)}
    \item Zostanie wywołane okienko z potwierdzeniem usunięcia, należy wpisać nazwę repozytorium w polu tekstowym i kliknąć \textbf{Confirm Deletion} \textit{Potwierdź usunięcie}
\end{itemize}

\subsubsection*{Oczekiwany rezultat}
Repozytorium wcześniej stworzone zniknęło z listy repozytoriów na naszym koncie.
\end{document}
