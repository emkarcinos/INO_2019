\documentclass{article}
\usepackage{polski}
\usepackage[utf8]{inputenc}
\usepackage{hyperref}
\hypersetup{
    colorlinks,
    citecolor=black,
    filecolor=black,
    linkcolor=black,
    urlcolor=black
}

\title{User Stories dla projektu "MOP Management"}
\author{Marcin Kostrzewski, Mateusz Tylka}
\date{22 Października, 2019r}

\begin{document}
\maketitle
\newpage
\tableofcontents
\newpage

\section{Szef}
Właściciel firmy sprecyzował nastsępujące scenariusze:
\begin{itemize}
    \item Logowanie do systemu z najwyższymi uprawnieniami, aby zarządzać kontami pracowniczymi
    \item Możliwość odzyskania zapomnianego hasła, w celu łatwego odzyskania dostępu bez ingerencji administatora
    \item Zmiana hasła do konta, aby konto było bezpieczne
    \item Tworzenie nowych projektów, aby móc rozpocząc pracę nad projektem
    \item Przydzielanie pracowników do projektów, aby organizować zespół
    \item Zmiana statusu projektu, aby przechowywać informacje o ukończeniu bądź innym stanie projektu
    \item Tworzenie zadań i rozdzielanie ich między pracownikami, aby każdy z nich wiedział o swoich obowiązkach
    \item Tworzenie planu działania, w celu łatwiejszej manipulacji czasem pracy każdego pracownika
    \item Wizualny raport o postępach w projektach, aby w prosty sposób dowiedzieć się o ewentualnych problemach, lub dostrzec postępy
    \item Informacje o statusie pracowników, aby wiedzieć, czy można przypisać im pracę
    \item Dodawanie danych klienta do projektu, żeby było jasne, dla kogo tworzony jest projekt
    \item Pozyskiwanie danych o kliencie z bazy, aby móc łatwo się z nim skontaktować
    \item Dostęp do kalendarza, tworzenie eventów, aby zarządzać czasem całego zespołu i mieć łatwy dostęp do najbliższych wydarzeń
    \item Wgląd do raportów finansowych, aby analizować wydatki
    \item Statusy poszczególnych sprzętów, żeby stwierdzać ewentualną wymianę i dbać o porządek
\end{itemize}

\newpage
\section{Administrator}
Administrator sporządził swoje wymagania poprzez następujące kryteria:
\begin{itemize}
    \item Logowanie do systemu przez aplikację terminalową, aby uzyskać dostęp do serwera
    \item Dostęp i logowanie przez SSH zdalnie, żeby móc w razie awarii zainterweniować bez konieczności podrózy do firmy
    \item Definiowanie na jakim porcie działa aplikacja, ponieważ na serwerze działa wiele programów i niektóre porty mogą być niedostępne
    \item Integralność danych w przypadku awarii, żeby firma nie straciła ważnych dla niej danych
    \item Tryb bezpieczny w przypadku nieoczekiwanego zamknięcia systemu, żeby poznać przyczynę awarii i zadecydować o dalszym działaniu programu
    \item Tworzenie i zarządzanie kontami użytkowników, aby nowi pracownicy mogli otrzymać nowe konta
    \item Przydzielanie uprawnień, w przypadku gdy obecny już pracownik zmieni swoje stanowisko
    \item Aktualizować system, aby szybko i łatwo dostarczyć dla użytkowników najnowszą wersję systemu
    \item Proste zamykanie systemu z odpowiednim zachowaniem, żeby proces nie zajmował wiele czasu i zapewniona była integralność danych
    \item Monitorowanie połączeń z serwerem, aby w razie ataków móc podjąć odpowiednie działania zabezpieczające system
    \item Historia logowania, aby w razie wewnętrznych konflików, czy nawet ataków z sieci mieć 'dowody zbrodni'
    \item Utrzymywanie kopii zapasowych, żeby w razie awarii sprzętowej móc odzyskać dane bezproblemowo
    \item Zoptymalizowaną i efektywną bazę danych, aby zaoszczędzić miejsce w przestrzeni dyskowej
    \item Dostęp do konta testowego, aby sprawdzać poprawność działania systemu w razie awarii, czy po aktualizacjach
    \item Dostęp do dokumentacji, aby móc idektyfikować problemy na podstawie kodów błędów i rozumieć zasady aktualizacji
    \item W przypadku nieprawidłowego działania systemu i niemożliwości zdiagnozowania problemu, możliwość kontaktu telefonicznego i szybkie rozwiązanie problemu przez firmę wykonującą, aby w krytycznych warunkach umożliwić szybko dostęp do aplikacji
\end{itemize}

\newpage
\section{Pracownik}
Pracownik chce:
\begin{itemize}
    \item Logować się do systemu przez interfejs webowy, aby móc korzystać z systemu na swoim koncie
    \item Mieć możliwość odzyskania hasła, którego zapomniał, żeby móc szybko powrócić do swobodnego korzystania z aplikacji
    \item Mieć możliwość zmiany hasła, aby dobrze zabezpieczać swoje konto
    \item Móc się wylogować z systemu, żeby jego konto było bezpieczne
    \item Dołączać do projektu, żeby pozostali członkowie firmy wiedzieli że pracuje właśnie nad tym konkretnym problemem
    \item Dobierać sobie zadania z tablicy, aby efektywnie realizować projekty i dzielić się obowiązkami z innymi pracownikami
    \item Mieć dostęp do planu działania, żeby organizować sobie pracę zgodnie z wizją szefa i innych członków zespołu
    \item Widzieć status projektu i jego zmiany, aby wiedzieć w jakiej fazie realizacji projekt się znajduje
    \item Móc generować raporty, aby zwizualizować swoim klientom, szefowi lub kolegom z zespołu postępy, bądź informacje o konkretnym projekcie
    \item Zmieniać swój status, aby udostępniać informacje o swojej obecnej dyspozycji
    \item Mieć podgląd do danych klientów, aby mieć możliwość do ewentualnych konsultacji z klientem w celu sprecyzowania wymagań i potrzeb klienta
    \item Mieć dostęp do kalendarza, aby móc odpowiednio gospodarować czasem wykonywania poszczególnych zadań
    \item Posiadać informacje o sprzęcie, żeby wiedzieć, czy może użyć go w danym projekcie
    \item Móc rejestrować czas swojej pracy, aby poprawnie rozliczyć się z szefem
    \item Posiadać możliwość zmiany swoich danych osobowych, aby zaaktualizować swoje informacje, na przykład o miejscu zamieszkania, czy o adresie mailowym
\end{itemize}

\end{document}