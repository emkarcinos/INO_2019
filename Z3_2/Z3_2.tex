\documentclass{article}
\usepackage{polski}
\usepackage[utf8]{inputenc}
\usepackage{hyperref}
\hypersetup{
    colorlinks,
    citecolor=black,
    filecolor=black,
    linkcolor=black,
    urlcolor=black
}

\title{User Stories dla projektu "MOP Management"}
\author{Marcin Kostrzewski, Mateusz Tylka}
\date{22 Października, 2019r}

\begin{document}
\maketitle
\newpage
\tableofcontents
\newpage

\section{Szef}
Właściciel firmy sprecyzował nastsępujące scenariusze:
\begin{itemize}
    \item Logowanie do systemu z najwyższymi uprawnieniami, aby zarządzać kontami pracowniczymi
    \item Możliwość odzyskania zapomnianego hasła, w celu łatwego odzyskania dostępu bez ingerencji administatora
    \item Zmiana hasła do konta, aby konto było bezpieczne
    \item Tworzenie nowych projektów, aby móc rozpocząc pracę nad projektem
    \item Przydzielanie pracowników do projektów, aby organizować zespół
    \item Zmiana statusu projektu, aby przechowywać informacje o ukończeniu bądź innym stanie projektu
    \item Tworzenie zadań i rozdzielanie ich między pracownikami, aby każdy z nich wiedział o swoich obowiązkach
    \item Tworzenie planu działania, w celu łatwiejszą manipulacją czasu pracy każdego pracownika
    \item Wizualny raport o postępach w projektach, aby w prosty sposób dowiedzieć się o ewentualnych problemach, lub dostrzec postępy
    \item Informacje o statusie pracowników, aby wiedzieć, czy można przypisać im pracę
    \item Dodawanie danych klienta do projektu, żeby było jasne, dla kogo tworzony jest projekt
    \item Pozyskiwanie danych o kliencie z bazy, aby móc łatwo się z nim skontaktować
    \item Dostęp do kalendarza, tworzenie eventów, aby zarządzać czasem całego zespołu i mieć łatwy dostęp do najbliższych wydarzeń
    \item Wgląd do raportów finansowych, aby analizować wydatki
    \item Statusy poszczególnych sprzętów, żeby stwierdzać ewentualną wymianę i dbać o porządek
\end{itemize}

\newpage
\section{Administrator}
\begin{itemize}
    \item Logowanie do systemu przez aplikację terminalową, aby uzyskać dostęp do serwera
    \item Dostęp i logowanie przez SSH zdalnie, żeby móc w razie awarii zainterweniować bez konieczności podrózy do firmy
    \item Definiowanie na jakim porcie działa aplikacja, ponieważ na serwerze działa wiele programów i niektóre porty mogą być niedostępne
    \item Integralność danych w przypadku awarii, żeby firma nie straciła ważnych dla niej danych
    \item Tryb bezpieczny w przypadku nieoczekiwanego zamknięcia systemu, żeby poznać przyczynę awarii i zadecydować o dalszym działaniu programu
    \item Tworzenie i zarządzanie kontami użytkowników, aby nowi pracownicy mogli otrzymać nowe konta
    \item Przydzielanie uprawnień, w przypadku gdy obecny już pracownik zmieni swoje stanowisko
    \item Aktualizowanie systemu, żeby 
    \item Proste zamykanie systemu z odpowiednim zachowaniem
    \item Monitorowanie połączeń z serwerem
    \item Historia logowania
    \item 
\end{itemize}



\end{document}