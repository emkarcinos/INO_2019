\documentclass{article}
\usepackage{polski}
\usepackage[utf8]{inputenc}
\usepackage[hidelinks]{hyperref}
\usepackage{forest}

\definecolor{folderbg}{RGB}{124,166,198}
\definecolor{folderborder}{RGB}{110,144,169}

\def\Size{4pt}
\tikzset{
  folder/.pic={
    \filldraw[draw=folderborder,top color=folderbg!50,bottom color=folderbg]
      (-1.05*\Size,0.2\Size+5pt) rectangle ++(.75*\Size,-0.2\Size-5pt);  
    \filldraw[draw=folderborder,top color=folderbg!50,bottom color=folderbg]
      (-1.15*\Size,-\Size) rectangle (1.15*\Size,\Size);
  }
}

\title{Obieg dokumentów w projekcie}
\author{Marcin Kostrzewski}
\date{19 Października, 2019r}

\begin{document}
    
\maketitle

\newpage
\tableofcontents

\newpage
\section{Wprowadzenie}
Celem tego dokumentu jest usystematyzowanie pracy z dokumentacją
w projekcie, omówienie struktury plików, zasad nazewnictwa,
zarządzaniem wersjami czy określenie zasad dostępu.

\section{Przechowywanie dokumentów}
Wszystkie dokumenty dotyczące projektu będą przechowywane na \textbf{Dysku Google}
(\url{drive.google.com}).
Aby umożliwić korzystanie z serwisu, zostało utworzone konto \textbf{Google}
na rzecz firmy. Dostęp do samego konta uzyskuje jedynie administrator, lecz każdy
pracownik ma dostęp do głównego folderu, w którym będą umieszczane wszystkie dokumenty.
Zostaną rozesłane wiadomości e-mail do każdego pracownika z prywatnym linkiem do 
całego folderu.

\section{Struktura katalogów}
Aby efektownie zbierać wszystkie dokumenty, usystematyzowana została struktura folderów,
która zapewni porządek w każdym projekcie. Wygląda ona następująco:
\newline

\begin{forest}
    for tree={
      font=\ttfamily,
      grow'=0,
      child anchor=west,
      parent anchor=south,
      anchor=west,
      calign=first,
      inner xsep=7pt,
      edge path={
        \noexpand\path [draw, \forestoption{edge}]
        (!u.south west) +(7.5pt,0) |- (.child anchor) pic {folder} \forestoption{edge label};
      },
      before typesetting nodes={
        if n=1
          {insert before={[,phantom]}}
          {}
      },
      fit=band,
      before computing xy={l=15pt},
    } 
  [Home
    [Project
        [Public
            [src]
            [export]
            [old]
            [queue]
            [draft]
        ]
        [Private
            [src]
            [export]
            [old]
            [queue]
            [draft]
        ]
    ]
  ]
\end{forest}
\newpage
\textbf{Legenda:}
\begin{itemize}
    \item Project: Nazwa projektu
    \item Public: Dokumenty, które mają zostać upublicznione
    \item Private: Dokumenty przeznaczone tylko dla pracowników
    \item src: Pliki źródłowe (edytowalne)
    \item export: Katalog z finalnymi, nieedytowalnymi dokumentami
    \item old: Poprzednie wersje dokumentów (tylko w formacie edytowalnym)
    \item queue: Dokumenty do akceptacji (nieedytowalne)
    \item draft: Katalog roboczy z plikami źródłowymi
\end{itemize}
\textbf{Uwaga!}
\newline
Jeżeli dany dokument zawiera dodatkowe materiały, takie jak obrazy, wykresy, czy inne
pliki, w katalogach \textit{src} i \textit{draft} powinien zostać utworzony \textbf{osobny folder} dla danego
dokumentu zawierający edytowalne pliki dokumentu oraz załączników.

\section{Uprawnienia}
\textbf{Google Drive} udostępnia dwa poziomy uprawnień: zapis i odczyt.
\subsection{Odczyt}
Domyślnie, każdy pracownik ma możliwość odczytu wszystkich dokumentów we wszystkich folderach, tj
do wszystkich projektów, do dokumentów publicznych i prywatnych i do wszystkich podkatalogów.
\subsection{Zapis}
Dostęp do absolutnego zapisu we wszystkich katalogach i podkatalogach dla danego projektu
przyznawany jest jedynie kierownikowi ds. dokumentacji oraz kierownikowi projektu.
Programiści i edytorzy mają możliwość zapisywania plików jedynie w folderze \textit{Private} dla danego
projektu, w którym pracują. Edytorzy dodatkowo mają dostęp do dokumentów publicznych. Mimo tego grupy te \textbf{nie powinne} umieszczać
czy edytować plików w folderach innych niż \textit{queue} i \textit{draft} bez upoważnienia przez
kierownika ds. dokumentacji lub kierownika projektu.

\section{Nazewnictwo plików}
W nazwach plików należy stosować się do szeregu ściśle określonych zasad:
\begin{itemize}
    \item Nie mogą występować żadne znaki białe
    \item Polskie znaki są zalecane
    \item Nie wolno używać symboli: \textbackslash / : * ? " \textless \textgreater
    \item Nazwa pliku powinna rozpoczynać się z wielkiej litery, i każde kolejne słowo powinno być rozpoczęte z wielkiej litery
    \item Musi zawierać numer wersji w formacie vV.R (numer wersji i rewizii)
\end{itemize}
Ogólnie:
\begin{center}
    NazwaPrzykładowegoPlikuv2.3.pdf
\end{center}

\section{Styl i formatowanie dokumentu}
Każdy dokument powinien posiadać:
\begin{itemize}
    \item Stronę tytułową zawierającą nazwę dokumentu, imię i nazwisko twórcy oraz datę pierwszego utworzenia
    \item Tabelę zmian w odniesieniu do wersji poprzedniej
    \item Spis treści
    \item Numer strony
\end{itemize}
Dodatkowo, wprowadzone zostały zasady dotyczące estetyki i formatowania:
\begin{itemize}
    \item Tekst ma być odpowiednio podzielony na paragrafy
    \item Justowanie tekstu jest obowiązkowe
    \item Dowolna czcionka typu \textbf{serif}
    \item Odniesienia do nazw funkcji kodu źródłowego, jego fragmenty i inne elementy powinny wyraźnie wyróżniać się od reszty tekstu, możliwie czcionką typu \textbf{monospaced}
    \item Stylizacja i formatowanie zgodne ze standardami \textbf{LaTeX} są zalecene
\end{itemize}

\section{Obieg}
\begin{center}
    \textit{Formuła opisana niżej jest jedynie uogólnieniem i może się różnić w zależności od projektu.}
\end{center}

Dokument jest stale rozwijany przez twórce w folderze roboczym \textit{draft} w danym projekcie. Po ukończeniu
danej wersji, dokument jest eksportowany do finalnego formatu i umieszczany w katalogu \textit{queue} przez twórcę dokumentu.
\par
Kierownik ds. dokumentacji w wyznaczonych do tego godzinach sprawdza foldery \textit{draft}
i przegląda dokumenty. Po dokładnej ocenie wysyła wiadomość zwrotną do twórcy
dokumentu. Jeżeli dokument spełnia wszystkie kryteria, zostaje zatwierdzony. Kierownik
upoważnia twórcę dokumentu do umieszczenia dokumentu w folderach \textit{src} i \textit{export}.
Jeżeli istnieją, usuwa poprzednie wersje z folderu \textit{src} i z folderu \textit{export} przenosi
dokument do folderu \textit{old}. W przypadku niezgodności dokumentu, odsyłana
jest wiadomość zwrotna z ewentualnymi poprawkami.
\par
Kierownik projektu po konsultacjach z klientem, podejmuje decyzje o upublicznieniu
danych dokumentów. Jeżeli dokumenty wymagają edycji, kierownik ds. dokumentacji
wyznacza edytora do danego dokumentu i przenosi go do folderu \textit{draft} znajdującego
się w folderze z dokumentami publicznymi dla danego projektu, oraz usuwa dokument z folderu \textit{queue}
\par
Edytor przydzielony do danego dokumentu pracuje nad nim w folderze \textit{draft}.
Po ukończeniu prac, eksportuje dokument do foleru \textit{queue} i oczekuje na
wiadomość zwrotną. Proces dalej wygląda tak samo jak w przypadku dokumentów prywanych.
\par
Przed ostatecznym upublicznieniem dokumentów klientowi, kierownik projektu
analizuje dokumenty z folderu \textit{export}. Jeżeli dokument nie wymaga 
poprawek, wysyłany jest klientowi. Jeżeli takie poprawki jednak istnieją, sprawa
trafia z powrotem do kierownika ds. dokumentacji.

\end{document}

