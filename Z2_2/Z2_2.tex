\documentclass{article}
\usepackage{polski}
\usepackage[utf8]{inputenc}
\usepackage{hyperref}
\hypersetup{
    colorlinks,
    citecolor=black,
    filecolor=black,
    linkcolor=black,
    urlcolor=black
}

\title{Wizja systemu informatycznego dla firmy produkującej filmy wideo MOP Media}
\author{Marcin Kostrzewski, Mateusz Tylka}
\date{15 Września, 2019r}

\begin{document}

\maketitle
\newpage
\tableofcontents
\newpage

\section{Wstęp}
Celem tego dokumentu jest analiza, zebranie, skompilowanie wszystkich
potrzeb i celów systemu "MOP Management". Przedstawione są w nim wymagania klienta, oraz sposob  w jaki system ma je realizować.
\section{Pozycjonowanie}
\subsection{Opis problemu}
\begin{itemize}
    \item Problemem jest słaba organizacja formalności firmy
    \item Skutkiem tego jest wolna, często nieefektywna praca
    \item Rozwiązanie tego problemu powinno polegać na znacznym usprawnieniu organizacji firmy
\end{itemize}
\subsection{Informacja o produkcie}
\begin{itemize}
    \item Klient: \textbf{MOP Media}
    \item Nazwa produktu: \textbf{MOP Management}
    \item Produkt ma usprawnić działanie firmy
\end{itemize}
\section{Opis klienta}
\subsection{Demografia}
Pracownicy w firmie klienta to osoby w wieku 20-30 lat.
Osoby te są biegłymi użytkownikami sprzętu komputerowego, korzystają
głownie z programów do edycji wideo i Excela, natomiast mają małą
styczność z programowaniem, oraz z niskopoziomowym działaniem systemów
informatycznych. W firmie pracuje również administrator zarządzający
serwerem, biegle porusza się w sprawach sieciowych, działanie systemów
informatycznych nie jest mu obce.
\subsection{Środowisko pracy klienta}
\begin{itemize}
    \item W firmie pracuje stale zatrudnionych sześciu pracowników, w tym jeden administrator
    \item Pracownicy pracują na komputerach zainstalowanych w budynku firmy. Nie wykonują pracy poza tym miejscem.
    \item Administrator pracuje zdalnie i zarządza serwerem firmy
\end{itemize}
\subsection{Kluczowe potrzeby użytkowników}
Użytkownicy do tej pory korzystali z programu \textit{Microsoft Office Excel} i nie
byli zadowoleni z systemu pracy ograniczonego przez program.
Poniżej lista podstawowych problemów związanych z obecnym systemem:
\begin{itemize}
    \item Duży poziom skomplikowania
    \item Konieczność znajomości zaawansowanych funkcji programu
    \item Brak formalizacji pracy
    \item Czasochłonne wprowadzanie danych i częste problemy z ich integralnością
\end{itemize}
Ponad to administator systemu sformułował poniższe proglemy:
\begin{itemize}
    \item Brak konkretnego sposobu przechowywania plików; surowe zasoby Excela trzymane są bezpośrednio na dysku w serwerze
    \item Skomplikowane wersjonowanie zajmujące dużą przestrzeń dyskową.
\end{itemize} 
Użytkownicy wzraz z administratorem chcieliby, aby wszystkie te problemy były rozwiązane.
\subsection{Alternatywy dla produktu}
Klient rozważał następujące alternatywy:
\begin{itemize}
    \item \textbf{GitHub projects}
    \item \textbf{Google Issue Tracker}
\end{itemize}
Powyższe propozycje zostały odrzucone, ponieważ oczekiwania przewidują
program specjalizujący się wyłącznie w dziedzinie produkcji wideo wraz z 
dodatkowymi funkcjami, które żadna z alternatyw nie oferuje.
\newpage
\section{Opis produktu}
\subsection{Perspektywa}
System ten będzie całkowicie niezależny i będzie tworzył jedną, integralną całość.
Nie będą z niego korzystały żadne zewnętrzne narzędzia ani systemy.
System ma zakładać możliwość dalszego rozwoju i implementacji nowych funkcji.
\subsection{Podsumowanie możliwości}
\begin{itemize}
    \item Stworzenie bazy danych: rozwiąże obecne problemy z integralnością danych, zapewni jej systematyzację oraz wyeliminuje problem wersji
    \item Interfejs: przyśpieszy pracę i zapewni łatwy dostęp do konkretnych fukncji systemu
    \item Generowanie raportów: udostępni łatwą wizualizację o procesie projektów klienta
    \item System zadaniowy: łatwiejsze rozdzielanie obowiązków między pracownikami
\end{itemize}
\subsection{Założenia}
Tworząc ten dokument założyliśmy, że na serwerze klienta na którym będzie zainstalowane nasze oprogramowanie
działa system operacyjny \textbf{Linux} z wersją jądra \textit{4.0 lub wyższej}. Sieć lokalna w firmie klienta działa poprawnie.
Każdy z komputerów w firmie klienta ma dostęp do sieci. Administrator uzyskuje dostęp do serwera przez protokół \textit{SSH}. 
\subsection{Dodatkowe koszta}
Przewidywane przez nas działanie systemu nie zakłada dodatkowych kosztów na rzecz sprzętu;
klient posiada już komputery klienckie, serwer i sieć LAN.
\subsection{Licencja i instalacja}
Produkt będzie bazował na zamkniętej licencjii, tzn. klienci nie będą mieli dostępu do kodu źródłowego.
Instalacja zostanie przeprowadzona przez wykonawców projektu, razem z administatorem sieci klienta.
\newpage
\section{Funkcje produktu}
\subsection{Baza danych}
W celu sformalizowania i usystematyzowania danych całe oprogramowanie będzie oparte
o kompleksową bazę danych. Baza ta będzie reprezentować następujące dane;
\begin{itemize}
    \item Klienci; informacje o klientach potrzebne do kontaktów, itp
    \item Projekty; spis prowadzonych projektów wraz z niezbędnymi informacjami
    \item Sprzęt; dane o wykorzystaniu sprzętu w konkretnych projektach oraz jego dostępność
    \item Rachunki; rozliczenia i finanse firmy
    \item Pracownicy; informacje o pracy pracowników oraz ich dane
\end{itemize}
\subsection{Interfejs}
Aby zapewnić łatwy dostęp do systemu dostępny będzie interfejs WEB dostępny dla każdego pracownika:
\begin{itemize}
    \item Prosty dostęp do bazy danych; warstwa abstrakcjii zapewni członkom szybki dostęp do wyszukanych przez nich informacji
    \item Wizualizacja danych; czytelna prezentacja wykresów, tabelek
    \item Interfejs do zarządzania zadaniami i śledzenie, postępów; zespół będzie miał możliwość gospodarowania czasem i śledzenia postępów w projekcie, odpowiednio rozdzielając między sobą zadania
    \item Generowanie raportów; raporty mogą być użyte w celach analitycznych,czy do dzielenia się postępem nad projektem projektu z klientami
\end{itemize}
\section{Ograniczenia}
\subsection{RODO}
Informacje o klientach w bazach danych powinny być odpowiednio zabezpieczone, tak aby były zgodne z \textbf{ustawami o ochronie danych osobowych RODO}.
\subsection{Brak dostępu do systemu z zewnątrz}
Należy pamiętać o tym, że system będzie postawiony w zamkniętej sieci LAN w biurze klienta i dostęp do niej będzie miał jedynie administrator poprzez \textit{SSH}.
\section{Jakość produktu}
\subsection{Interfejs webowy}
Wygląd interfejsu będzie oparty o płaską kolorystykę, minimalizm i czytelność.
\subsection{Baza danych}
Nie można dopuścić do utraty danych, ponieważ może to skutkować odpowiedzialnością karną dla firmy klienta.
Wyciek danych będzie pogwałceniem regulacji prawnych RODO.
\subsection{Ogólne działanie systemu}
Cały system powinien działać szybko i sprawnie.
\section{Priorytety}
Najważniejszym elementem w tym systemie jest baza danych. Model danych powinien być starannie przemyślany
i wielokrotnie przedyskutowany z klientem. Trzeba też położyć duży nacisk na bezpieczeństwo danych.
Najmniejszym priorytetem są raporty.
\section{Inne wymagania produktowe}
\subsection{Standardy}
\begin{itemize}
    \item Interfejs webowy oparty o semantyczny HTML5 i CSS
    \item Komunikacja serwer - klient za pomocą protokołu TCP/IP
    \item Baza danych i serwer postawiony na systemie Linux
    \item Bezpieczeństwo bazy danych zgodne z przepisami RODO
\end{itemize}
\subsection{Wymagania systemowe (serwer)}
\begin{itemize}
    \item Czterordzeniowy procesor Intel lub AMD, 2GHz
    \item 4GB RAM
    \item 1TB przestrzeni dyskowej
\end{itemize}
\section{Dokumentacja}
\subsection{Dokumentacja serwerowa}
Będą w niej zawarte informacje o bazie danych, o zasadzie działania aplikacji serwerowej, zachowaniu 
w przypadkach awarii i kodach błędów.
\subsection{Pomoc dla użytkownika}
W aplikacji webowej będzie dostępna sekcja z pomocą opisująca działanie konkretnych funkcji interfejsu.
\end{document}
