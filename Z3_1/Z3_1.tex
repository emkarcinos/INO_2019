\documentclass{article}
\usepackage{polski}
\usepackage[utf8]{inputenc}
\usepackage{hyperref}
\hypersetup{
    colorlinks,
    citecolor=black,
    filecolor=black,
    linkcolor=black,
    urlcolor=black
}

\title{Zakres systemu "MOP Management"}
\author{Marcin Kostrzewski, Mateusz Tylka}
\date{22 Października, 2019r}

\begin{document}

\maketitle
\newpage
\tableofcontents
\newpage

\section{Wprowadzanie}
Celem tego dokumentu jest sprecyzowanie aktorów, czyli ról różnych
użytkowników w systemie i ich opis, oraz sprecyzowanie konkretnych 
wymagań zakresu systemu "MOP Management". Dokument został stworzony
zgodnie z wizją projektu opisaną osobno.

\section{Role}
W projekcie sprecyzowane zostały następujące role:
\begin{itemize}
    \item \textbf{Właściciel firmy, szef} - Tworzy, zarządza projektami, pracownikami, organizacją oraz finansami firmy, generuje raporty, decyduje o ewentualnym rozwoju systemu
    \item \textbf{Administrator} - Zarządza serwerem, technicznym aspektem systemu, utrzymuje bazę danych i sieć lokalną, wdraża aktualizacje
    \item \textbf{Pracownik} - Sprwadza i wykonuje powierzone mu zadania, korzysta z kalendarza i zarządza sprzętem w posiadaniu firmy
    \item \textbf{Urzędnik} - Analizuje finanse, ma wgląd w raporty firmy
\end{itemize}

\section{Zakres systemu}
Aby zdefiniować zakres funkcjonalności systemu, sporządzona została tzw. \textit{Tabela IN/OUT}.
Opisuje konieczność implementacji konkretnych funkcji a(\textit{IN}), opcjonalność (\textit{OPT}), prezycuje działania,
które nie są objęte w ramach projektu (\textit{OUT}).
\begin{center}
    \begin{tabular}{| p{2cm} | p{3cm} | p{4cm} | p{2cm} |}
    \hline
    Funkcja & IN & OUT & OPT \\
    \hline
    \hline
    Dostęp & Sieć lokalna firmy, Do serwera przez SSH & Dostęp zdalny do interfejsu & \\
    \hline
    Autoryzacja & Panel logowania, konta, grupy, odzyskiwanie hasła, 'zapamiętaj mnie', tworzenie kont & Weryfikacja dwuetapowa, logowanie przez strony zewnetrzne (Facebook, Google) & Automatyczne logowanie\\
    \hline
    Zarządznie zadaniami & Rozdzielanie zadań między pracownikami, zarządzanie czasem i harmonogramem & Automatyczne przydzielanie zadań & Czat \\
    \hline

    \end{tabular}
\end{center}
\begin{center}
    \begin{tabular}{| p{2cm} | p{3cm} | p{4cm} | p{2cm} |}
    \hline
    Funkcja & IN & OUT & OPT \\
    \hline
    \hline
    Interfejs & WEB, HTML, kalendarz & Aplikacja terminalowa, niezależny program, wykrywanie błędów w tekście, interfejs mobilny & Text to speech, motywy \\
    \hline
    Baza danych & Zarządzanie, wizualizacja, wprowadzanie danych, szyfrowanie haseł & Modyfikacja struktury, przechowywanie plików, dostęp do skryptów SQL tworzących bazę &  Interfejs SQL \\
    \hline
    Raporty & Generowanie PDF, wykresy & Arkusze kalkulacyjne, pliki tekstowe, prognozy & Markdown \\
    \hline
    Aktualizacje & Skompilowane pliki aktualizacyjne & Kod źródłowy, powrót do poprzedniej wersji, OTA, aktualizacje automatyczne  & Pluginy \\
    \hline
    Serwer & Dystrybucja na systemy Linux, rodzina Debian, dostęp przez SSH & Dystrybucja na systemy Windows, MacOS, interfejs graficzny & Inne dystrybucje Linux\\
    \hline
    Finanse & Faktury PROForma, notacja wykładnicza & Paragony, konwersja walut, skarbonka, przechowywanie fizycznych pieniędzy, bankowość, płatności elektroniczne & \\
    \hline
    Sprzęt & & Sprawdzenie wartości sprzętu, wycena projektów ze względu na sprzęt, automatyczny wynajem, automatyczne zakupy, & \\
    \hline
    Zarządzanie projektem & Wprowadzanie, zmiana statusu, notatki, przydzielanie pracownikom & Usuwanie projektów z historii, automatyczna wycena & \\
    \hline
    Kaledarz & Notatki & Przypomnienia, alarmy, załączanie plików & \\
    \hline
    SSH & Dostęp spoza sieci lokalnej & Interfejs graficzny, tworzenie kont, FTP& \\
    \hline
    Dane pracowników & Dane personalne, zarobki, siatka godzin & Hasła, zewnętrzne pliki, rekordy kryminalne, poprzedni pracownicy & \\
    \hline
    Dane klientów & Numer telefonu & Adres, dowody, paszporty, PESEL, zdjęcia & \\
    \hline


    \end{tabular}
\end{center}
\end{document}