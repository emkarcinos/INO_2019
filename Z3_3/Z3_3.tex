\documentclass{article}
\usepackage{polski}
\usepackage[utf8]{inputenc}
\usepackage{hyperref}
\hypersetup{
    colorlinks,
    citecolor=black,
    filecolor=black,
    linkcolor=black,
    urlcolor=black
}

\title{Use Case dla projektu "MOP Management"}
\author{Marcin Kostrzewski}
\date{22 Października, 2019r}

\begin{document}

\maketitle
\newpage
\tableofcontents
\newpage

\section{Wprowadzenie}
Dokument ten opisuje use case w projekcie "MOP Management" \textit{(Wizja projektu została opisana w innym pliku)}.
Skupiać się będzie na jednym z przypadków użycia - \textbf{Dodanie nowego projektu}.
Pomoże to w późniejszym projektowaniu systemu i w implementacji danych funkcji.

\section{Use Case: Wprowadzenie projektu do systemu}
\textbf{Aktor podstawowy: \textit{Szef firmy}}

\subsection{Główni odbiorcy i oczekiwania względem systemu}
\begin{itemize}
    \item Szef: jest to podstawowa funkcja z jakiej szef będzie korzystał. Zależy mu na tym, aby dostęp do niej był szybki. Chce, aby wprowadzanie danych o projekcie było objęte pewnymi stałymi ramami, do których interfejs będzie dostosowany.
    \item Pracownik: oczekuje dostępu do nowego projektu wraz ze wszystkimi istotnymi informacjami o nim
    \item Klient: nie chciałby, aby musiał podawać dane kilkakrotnie, powinien mieć możliwość otrzymania informacji o stanie jego projektu
    \item Przedsiębiorstwo: potrzebuje dobrej organizacji wszystkich projektów, systematyzowania danych i ich bezpieczeństwa. W przypadku awarii systemu bądź sprzętu, musi mieć możliwość odzyskania krytycznych danych
\end{itemize}

\subsection{Warunki wstępne}
Szef jest zalogowany w systemie i ma dostęp do interfejsu zarządzania zadaniami. Baza danych ma wystarczającą ilość przestrzeni
dyskowej, aby mógł zostać dodany wpis o projekcie.

\subsection{Warunki końcowe}
Baza danych jest zaktualizowana o wpis z wszystkimi koniecznymi informacjami o projekcie. Ustalana 
jest data wprowadzenia projektu do systemu, generowany jest wpis w historii. Na osobnym dysku tworzona jest
kopia zapasowa z wpisem do bazy. Projekt jest widoczny na tablicy pracowników i tablicy szefa. System jest w 
gotowości do przydzielenia ról i zadań pracownikom. 

\subsection{Scenariusz główny}
\begin{enumerate}
    \item Podczas rozmowy z klientem, szef otwiera moduł tworzenia projektu na swoim komputerze PC poprzez aplikacę webową
    \item Szef wprowadza dane o projekcie, wypełnia konieczne pola, między innymi: nazwę projektu, typ, krótki opis i zatwierdza utworzenie.
    \item Po zatwiedzeniu wyświetlane są wprowadzone dane w zwizualizowanej postaci (tabela)
    \item Szef ma możliwość modyfikacji lub zatwierdzenia projektu \newline \textit{W przypadku modyfikacji, powtarza krok 2 i 3}
    \item System przekazuje dane z klienta do serwera. Tworzony jest nowy wpis w bazie danych oraz kopia zapasowa.
    \item Na aplikacjach klienckich wyświetla się powiadomienie o nowym projekcie.
    \item Szef otrzymuje wiadomość zwrotną, że projekt został dodany pomyślnie
    \item W dowolnym momencie może zostać wprowadzana zmiana w danych projektu, mogą zostać załączone zewnętrzne dokumenty. \textit{Powtarzane są wtedy kroki 2-4}. Wpis w bazie danych oraz kopia zapasowa są aktualizowane.
\end{enumerate}

\subsection{Rozszerzenia}

    \subsubsection*{* W dowolnym momencie system ulega awarii}
    Integralnośc danych musi zostać zachowana, zatem konieczna jest transakcyjność. Aktualizacje
    i tworzenie wpisów o projekcie musi zostać zaktualizowane tylko w przypadku kompletnego przesłania
    danych z aplikacji klienckiej do serwera i bazy danych
    \begin{enumerate}
        \item W przypadku automatycznego powrotu systemu do stanu gotowości, szef sprawdza, czy wpis o nowym projekcie został utworzony
    \begin{enumerate}
            \item W przypadku braku wpisu, proces dodania projektu jest powtarzany
            \item W przypadku poprawnego dodania wpisu, proces przebiega pomyślnie i szef może kontynuować pracę
        \end{enumerate}
        \item Gdy system nie powraca do stanu gotowości, szef musi poinformować administratora o błędzie w systemie. Administrator podejmuje odpowiednie kroki, aby przywrócić system do działania
    \end{enumerate}

    \subsubsection*{2. Niekompletne dane}
    W przypadku, gdy nie zostanie wprowadzona \textit{nazwa projektu}, projekt nie zostanie dodany, szef zostanie poinformowany o błędzie.

    \subsubsection*{5a. Błąd połączenia z serwerem lub błąd bazy danych}
    Wpis o nowym projekcie nie zostaje dodany. Aplikacja kliencka informuje o błędzie połączenia z serwerem i konieczności skontakowania się z administratorem

    \subsubsection*{5b. Projekt już istnieje}
    Gdy system bazodanowy wykryje, że projekt o tej samej nazwie istnieje już w bazie, wpis o nim nie jest dodawany. Interfejs sygnalizuje o błędzie i konieczności zmiany nazwy projektu.

    \subsubsection*{8a. Błędnie wprowadzone dane}
    Gdy system wykryje, że np. zmieniana jest nazwa projektu na nazwę już istniejącą, zwróci błąd.

    \subsubsection*{8b. Załączanie plików}
    Szef może chcieć dodać dokument do projektu. 
    \begin{enumerate}
        \item Interfejs graficzny oferuję przesłanie i tym samym załączenie pliku do projektu.
        \item Szef wybiera plik i przesyła go na serwer.
        \begin{enumerate}
            \item Obliczana jest ilość wolnego miejsca na dysku serwerowym. Gdy ilość wolnego miejsca nie jest wystarczająca do przesłania pliku, zostaje zwrócony błąd.
            \item Podczas przesyłania występuje zerwanie połączenia. Plik nie jest przesłany i zostaje wysłane powiadomienie do aplikacji klienckiej.
            \item Gdy plik jest wysłany poprawnie i znajduje się już na serwerze, tworzona jest jego kopia zapasowa, oraz dodawany jest wpis w bazie danych zawierający ścieżkę do dokumentu.
            \item W przypadku sukcesu wszystkich procesów, zwracana jest wiadomość o sukcesie
        \end{enumerate}
    \end{enumerate}

\subsection{Wymagania specjalne}
\begin{itemize}
    \item Szef chcę mieć możliwość dodawania nieograniczonej liczby nieskategoryzowanych notatek podczas tworzenia projektu
    \item W nazwie projektu mogą występować polskie znaki
    \item Załączane pliki mogą mieć rozmiar powyżej 4GB
\end{itemize}

\subsection{Wymagania technologiczne oraz ograniczenia na wprowadzane dane}
\begin{itemize}
    \item 2. Nazwa projektu nie może zawierać więcej niż 255 znaków
    \item 8b. Pliki będą załączane prosto z dysku na komputerze klienckim
    \item 8b. Prędkość przesyłu plików jest ograniczona przez sieć klienta
\end{itemize}

\subsection{Kwestie otwarte}
\begin{itemize}
    \item Czy projekt może nie mieć daty rozpoczęcia?
    \item Czy dowolny pracownik może z łatwością zrestartować serwer w przypadku awarii?
\end{itemize}
\end{document}
