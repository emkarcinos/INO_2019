\documentclass{article}
\usepackage{polski}
\usepackage[utf8]{inputenc}

\title{Pytania do Scrum'a}
\author{Marcin Kostrzewski}
\date{20 Października, 2019r}

\begin{document}

\maketitle
\newpage

\section{Obowiązki Scrum Mastera}
Scrum Master dba o przestrzeganie zasad zgodnych ze Scrumem, pomaga pracownikom
projektu zrozumienie idei działania w Scrumie, jest swego rodzaju
'coachem' dla zespołu deweloperskiego. Scrum Master pewni również rolę
'bufora' między drużyną a ludźmi spoza niej. Odpowiednio decyduję
co i jak przekazać drużynie, aby treść była zgodna z technikami
Scrumu.
\par
Scrum Master dba też o odpowiednie precyzowanie celów sprintów,
zarządza Backlogiem wraz z Product Ownerem, aby był odpowiednio
czytelny i zrozumiały. Pomaga również drużynie w organizacji tak,
aby byli w stanie tworzyć wysokiej jakości produkty.

\section{Retrospektywa}
Retrospektywa sprintu to comiesięczne spotkanie całej drużyny w celach
dyskusji nad obecnymi problemami, okazjami na ulepszenie pracy. Oceniana
jest praca całej drużyny oraz tworzone są dalsze plany działania
i rozwoju projektu.

\section{Definition of Done dla projektu}
\begin{center}
    \textit{W odniesieniu do projektu 'MOP Management' z poprzedniego zadania}
\end{center}
\textit{Definition of Done} oznaczałoby ukończenie kolejnej funkcjonalności dostępnej
dla użytkownika, tj. kolejnej funkcji interfejsu, która jest funkcjonalna,
spełnia zasady określone w wizji projektu i jest pozbawiona błędów, które
można skrajnie łatwo dostrzec. Co za tym idzie, z daną funkcją interfejsu
wiążę się ukończenie strony serwera realizujacą tą funkcję, czy dany
element bazy danych. Z każdym cotygodniowym sprintem użytkownik powinien
móc zobaczyć kolejną działającą funkcje w jego programie. Te bardziej skomplikowane
funkcje, opierające się na wielu dependencjach z projektu będą realizowane
w kolejności późniejszej.
\end{document}